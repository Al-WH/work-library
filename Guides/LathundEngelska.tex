\documentclass{article}
\usepackage[utf8]{inputenc}
\usepackage{fancyhdr}
\usepackage{graphicx}
\usepackage{geometry}

% ---- Commands ------- %
\newcommand{\documentNumber}[1]{
    \LARGE  \textbf{ PUSS2142{#1} } \\
    \medskip
}
\newcommand{\documentVersion}[1]{
    v. {#1}

    \medskip
}
\newcommand{\documentTitle}[1]{
    \centerline{\rule{13cm}{0.4pt}}
    \bigskip \bigskip
    \LARGE {#1} \\
    \bigskip \bigskip
    \centerline{\rule{13cm}{0.4pt}}
}
\newcommand{\documentGroup}[1]{
    \bigskip \bigskip
    \LARGE Group {#1} \\
    \bigskip
}
\newcommand{\documentResponsible}[1]{
    \LARGE Responsible: {Anna Bergvall} \\
    \medskip
}
\newcommand{\documentAuthors}[1]{
    \LARGE Authors: {Anna Bergvall} \\
    \medskip    
}
\newcommand{\documentDate}[1]{
    \date {#1} 
}

\graphicspath{{./images/}} % Defines a path to file images
\renewcommand{\arraystretch}{1.7}  % Vertical padding for tables


% --- Header & Footer ---- %
\pagestyle{fancy}
\lhead{\leftmark}
\rhead{}
\rfoot{\thepage}
\cfoot{}
\lfoot{}


% ------------------------------------------------ #

% ----- FILL THIS ----- %
\title {
    % Must be 2 digits
    \documentNumber {01}    
    
    % BASELINE.VERSION
    \documentVersion {0.1}
    
    % Full name - SHORTNAME
    \documentTitle {Writing in English - A Small Formatting Guide}
    \documentGroup {2}
    
    % Options: - Project Management Group
    %          - System Architecture Group
    %          - Developer Group
    %          - Test Group
    \documentResponsible {Project Management Group}
    \documentAuthors {Project Management group}
    
    % Format: YYYY-MM-DD
    \documentDate {2021-02-03}
}

\begin{document}

\maketitle
\thispagestyle{empty}

\newpage

\tableofcontents

\newpage

%---------Document begins here-----------

\section{Introduction}
The purpose of this text is to serve as a guide when writing the project documents. 

\section{Capital Letters... When?!}
\subsection{Headers}
The first letter of each word in a header should be capitalised \textbf{except} articles (\textit{a}, \textit{an}, \textit{the}, \textit{that}, and so on), short prepositions (e.g., \textit{on}, \textit{to}), conjunctions (like \textit{and}, \textit{or}, \textit{neither... nor}).
\begin{itemize}
    \item Rule of thumb: Capitalise words that have lexical meaning. Compare the words \textit{cat} and \textit{a}.
\end{itemize}

\subsection{Commas} 
Moreover, capitalise the first letter of the first word after a colon:

\begin{itemize}
\item Like this: There was more work to be done.
\end{itemize}

\subsection{Days and Months}
All weekdays and months begin with a capital letter:

\begin{itemize}
    \item The project plan is due Friday.
    \item See you in April.
\end{itemize}
\section{Is It \textit{It Is} or \textit{It's}?}
Contractions are not used in formal writing in English. One writes \textit{we are}, \textit{there is}, \textit{it is}, \textit{how is}, \textit{and so on}. 



\section{Punctuation and Quotes}

\subsection{Commas and Full Stops}
Commas and full stops (\textit{period} in American English) are put \textit{within} the quotation marks.

\begin{itemize}
    \item Christine says that "Group 2 is doing really well so far." 
    \item "That was really nice of her," she said.
\end{itemize}

\subsection{Colons and Semicolons}

Colons and semicolons are put \textit{outside} the quotation marks:

\begin{itemize}
    \item There were "many problems": First, there was the...
\end{itemize}

\subsubsection{The Oxford Comma}
I like apples, oranges, and pears. The last comma is called the Oxford Comma and is grammatically correct and should be used.

\section{American or British English?}

The project group have decided that British spelling and grammar be used in the project documents. Some important differences are listed below:

\subsection{Spelling}
\begin{tabular}{c|c|c}
    British Spelling & American Spelling \\
    \hline
    neighbour, behaviour, colour & neighbor, behavior, color \\
    cancelled, cancelling & canceled, canceling \\
    traveller, travelling & traveler, traveling \\
    theatre, centre & theater, center \\
    analyse, paralyse & analyze, paralyze \\

\end{tabular}


\subsection{Grammar}

    Collective nouns, like \textit{Project Group} or \textit{Manchester United}, take either a singular or a plural verb agreement in British English, whereas in American English they only take the singular agreement (they see the group as one unit). 
    
\begin{itemize}
    \item Example: Manchester United were okay today (BrE). 
     \item Example: The Development Team was excited to get going (AmE).
\end{itemize}

\subsection{Dates and Time}

\begin{tabular}{c|c}
    British English & American English \\
    \hline
    3 February 2021 & February 3, 2021 \textit{or} 3 February 2021 \\
    03/02/21 & 02/03/21 \\
\end{tabular}


\section{Looking Forward to See You... or Seeing You?}
When should the -ing form (the progressive form) be used? It is used after prepositions (to, from, on, and so on) and 
When the word or phrase following \textit{to} could be replaced with \textit{it} - use the -ing form:

\begin{itemize}
    \item I am looking forward to seeing you. 
     \newline \textit{Seeing you} can be replaced with \textit{it}. The correct form is \textit{seeing you} and not \textit{see you.}
    \item I am very excited to see you.
    \newline I am very excited to it does sound weird. Conclusion: \textit{See} is the correct verb form.
\end{itemize}

Note that the progressive verb form is always used after all forms of the verb \textit{be} (\textit{am, are, is}):

\begin{itemize}
    \item I am going to bed.
    \item He is singing.
\end{itemize}




%---------Document ends here-----------

\end{document}
