\documentclass{article}
\usepackage[utf8]{inputenc}
\usepackage{fancyhdr}
\usepackage{graphicx}
\usepackage{geometry}

% ---- Commands ------- %
\newcommand{\documentNumber}[1]{
    \LARGE  \textbf{ PUSS21420{#4} } \\
    \medskip
}
\newcommand{\documentVersion}[1]{
    v. {#1}

    \medskip
}
\newcommand{\documentTitle}[1]{
    \centerline{\rule{13cm}{0.4pt}}
    \bigskip \bigskip
    \LARGE \textbf{TimeMate} \\
    \bigskip
    \LARGE {#1} \\
    \bigskip \bigskip
    \centerline{\rule{13cm}{0.4pt}}
}
\newcommand{\documentGroup}[1]{
    \bigskip \bigskip
    \LARGE Group {#1} \\
    \bigskip
}
\newcommand{\documentResponsible}[1]{
    \LARGE Responsible: {#1} \\
    \medskip
}
\newcommand{\documentAuthors}[1]{
    \LARGE Authors: {#1} \\
    \medskip    
}
\newcommand{\documentDate}[1]{
    \date {#1} 
}

\graphicspath{{./images/}} % Defines a path to file images
\renewcommand{\arraystretch}{1.7}  % Vertical padding for tables


% --- Header & Footer ---- %
\pagestyle{fancy}
\lhead{\leftmark}
\rhead{}
\rfoot{\thepage}
\cfoot{}
\lfoot{}


% ------------------------------------------------ #

% ----- FILL THIS ----- %
\title {
    % Must be 2 digits
    \documentNumber {01}    
    
    % BASELINE.VERSION
    \documentVersion {0.1}
    
    % Full name - SHORTNAME
    \documentTitle {Software Top Level Design Document}
    \documentGroup {2}
    
    % Options: - Project Management Group
    %          - System Architecture Group
    %          - Developer Group
    %          - Test Group
    \documentResponsible {System Group}
    \documentAuthors {Developer Group}
    
    % Format: YYYY-MM-DD
    \documentDate {2021-02-11}
}

\begin{document}

\maketitle
\thispagestyle{empty}

\newpage

\tableofcontents

\newpage

%---------Document begins here-----------

% FILL IN CORRECT VERSION HISTORY!
% Not sure? Refer to SDP how it works or ask someone!
\section{Document History}
\begin{tabular}{ l | l | l | l }
    Version & Date & Responsible & Description \\
    \hline
    0.1 & 2021-02-11 & UG & Document created. \\
\end{tabular}

\section{Introduction}
This document describes the design of "TimeMate". The system is a further development of the system "BaseBlockSystem". "TimeMate" hosts various functions related to the creating of time reports, getting a view of reported time and more.

\section{Reference Documents}
\begin{enumerate}
    \item Software Requirements Specification: TimeMate, v. 0.5, Doc. number: PUSS214201
    \item Software Top Level Design Document: BaseBlockSystem, v 1.0, Doc. number: PUSS12004
\end{enumerate}

\section{Terminology}
\begin{itemize}
\item \textbf{Java Server Page (JSP):} Server-side technology that enables the creation of dynamic views.
\item \textbf{Servlets:} Java programs that runs on the server side.
\item \textbf{Java Beans:} A Java class that only contains set/get methods with private attributes
\end{itemize}

\section{Overview}
Placeholder for the overview of the project - replace with information regarding the project!

\subsection{Java Server Pages}
Below are the pages used for "TimeMate" and what their function is.

\subsubsection{login.jsp}
The view that is shown when a user that is not logged tries to access the system. This is the only page in the "TimeMate" system an user can access without having an account. The page consists of two fields, one for username and one for password, as well as an option to request a new password in case that the user has forgotten their password. If the user chooses to reset their password, a pop up is shown where the user can enter his/her mail which the new password will be sent to.

\subsubsection{index.jsp}
This file contains the view that is shown when an user has logged in. From here, the user can access the menu, which is dynamically updated depending on what group the user belong to. The index page contains overview information about TimeMate.

\subsubsection{viewReport.jsp}
This file contains the view that shows a collection of the users submitted reports. When here, the user can select a submitted report to show more information about.

\subsubsection{newReport.jsp}
This file contains the view that represents the creation of a new time report.

\subsubsection{editReport.jsp}
This view is used by users that want to edit their previous reported time reports. When here, the user can select a submitted report to edit.

\subsubsection{summaryReport.jsp}
This view represents a summary of a users previous reported time reports.

\subsubsection{signReport.jsp}
This view is used by project leaders for signing and unsigning time reports.

\subsubsection{usermanagement.jsp}
This file contains the view of the usermanagement page. The view is used by the administrator and the project leaders.

\subsubsection{administration.jsp}
This file contains the view of the administration page. Only the administrator has access to this view.

\subsubsection{changepassword.jsp}
This view is used by all users to change their password. The page contains three fields, one for the user current password, and two for the new password (in order to confirm it).

\subsection{Servlets}
The files below are the controllers for the system.

\subsubsection{LoginServlet}
Communication link between the view login.jsp and the bean LoginBean.

\subsubsection{PasswordChangerServlet}
Servlet used when a user changes their password.

\subsubsection{TimeReportServlet}
Communication link between the view report.jsp and the bean ReportBean.

\subsubsection{UserManagementServlet}
Communication link between the view usermanagement.jsp and the bean UserManagementBean.

\subsubsection{AdministrationServlet}
Communication link between the view administration.jsp and the bean AdministrationBean.

\subsection{Java Beans}
The files below are the Java Beans that are used to send information from the server to the client view. When created, beans are stored in the current session and discarded when the session ends.

\subsubsection{UserBean}
Get/set class which contains user information required to execute queries and updates related to the logged in user.

\subsubsection{TimeReportBean}
Get/set class which contains a list of users and their time reports which is required to render the report.jsp view.

\subsubsection{UserManagementBean}
Get/set class which contains a list of users(Excluding project leaders) and their roles which is required to render the usermanagement.jsp view.

\subsubsection{AdministrationBean}
Get/set class which contains a list of users and their roles (Including project leaders) which is required to render the administration.jsp view.

\subsection{Other classes}
The classes below are never accessed by the user in any way, but are instead helper classes to the servlets.

\subsubsection{MailHandler}
Class used to send e-mails to users upon account creation and password changes.

\subsubsection{PasswordHandler}
Class used to encrypt passwords when needed.

\subsubsection{Database}
Class to establish connection between Tomcat and our database.

\section{Database}
Diagram etc. for database here

\section{Information Stored in Sessions}

%-Slå ihop 0, 1, 2 eftersom de har samma privileges?
\textbf{Integer CURRENT\textunderscore ROLE:} Used to determine the current users privileges. The following states have been defined:
\medskip

\indent \textbf{0:} Privileges for users in the "UG" group.

\indent \textbf{1:} Privileges for users in the "TG" group.

\indent \textbf{2:} Privileges for users in the "SG" group.

\indent \textbf{3:} Project leader privileges.

\indent \textbf{4:} Administrator privileges.

\section{Sequence Diagrams}

\subsection{}

%---------Document ends here-----------

\end{document}