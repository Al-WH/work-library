\documentclass{article}
\usepackage[utf8]{inputenc}
\usepackage{fancyhdr}
\usepackage{graphicx}
\usepackage{geometry}
\usepackage{enumerate}

% ---- Commands ------- %
\newcommand{\documentNumber}[1]{
    \LARGE  \textbf{ PUSS2142{#1} } \\
    \medskip
}
\newcommand{\documentVersion}[1]{
    v. {#1}
    \medskip
}
\newcommand{\documentTitle}[1]{
    \centerline{\rule{13cm}{0.4pt}}
    \bigskip \bigskip
    \LARGE \textbf{TimeMate} \\
    \bigskip
    \LARGE {#1} \\
    \bigskip \bigskip
    \centerline{\rule{13cm}{0.4pt}}
}
\newcommand{\documentGroup}[1]{
    \bigskip \bigskip
    \LARGE Group {#1} 
    \bigskip
}
\newcommand{\documentResponsible}[1]{
    \LARGE Responsible: {#1} \\
    \medskip
}
\newcommand{\documentAuthors}[1]{
    \LARGE Authors: {#1} \\
    \medskip    
}
\newcommand{\documentDate}[1]{
    \date {#1} 
}

\graphicspath{{./images/}} % Defines a path to file images
\renewcommand{\arraystretch}{1.7}  % Vertical padding for tables


% --- Header & Footer ---- %
\pagestyle{fancy}
\lhead{\leftmark}
\rhead{}
\rfoot{\thepage}
\cfoot{}
\lfoot{}


% ------------------------------------------------ #

% ----- FILL THIS ----- %
\title {
    % Must be 2 digits
    \documentNumber {02}    
    
    % BASELINE.VERSION
    \documentVersion {0.5}
    
    % Full name - SHORTNAME
    \documentTitle {Software Verification and Validation Specification}
    \documentGroup {2}
    
    % Options: - Project Management Group
    %          - System Architecture Group
    %          - Developer Group
    %          - Test Group
    \documentResponsible {Test Group}
    \documentAuthors {Test Group}
    
    % Format: YYYY-MM-DD
    \documentDate {2021-02-10}
}

\begin{document}

\maketitle
\thispagestyle{empty}

\newpage

\tableofcontents

\newpage

%---------Document begins here-----------


\section{Document History}
\begin{tabular}{ l | l | l | l }
    Version & Date & Responsible & Description \\
    \hline
    0.1 & 2021-01-25 & TG & Document created. \\
    0.2 & 2021-02-02 & TG & Ready for informal review. \\
    0.3 & 2021-02-04 & TG & Corrected grammatical errors and typos. \\
    0.4 & 2021-02-10 & TG & Edited references, typos and rephrased tests, after formal review. \\
    0.5 & 2021-02-11 & TG & Corrected references, typos, and more, after second infromal review. \\

\end{tabular}

\section{Introduction}

	This document describes how the TimeMate system should be verified. 

\section{Reference Documents}

\begin{enumerate}
  \item Projekthandledning 
  \item Software Requirements Specification: TimeMate, v. 0.5, Doc. number: PUSSS214201
  \item Software Requirements Specification: BaseBlockSystem, v. 1.0, Doc. number: PUSS12002

\end{enumerate}

\section{Overview}
During the project, three formal reviews are carried out.


\begin{enumerate}
	\item Software Specification Review (SSR)
		\begin{enumerate}[a.] 
			\item SDP
			\item SRS 
			\item SVVS 
		\end{enumerate}
	\item Preliminary Design Review (PDR) 
		\begin{enumerate}[a.]
			\item SVVI
			\item STLDD
		\end{enumerate}
	\item Product Review (PR) 
		\begin{enumerate}[a.]
			\item SVVR 
			\item SSD 
			\item PFR 
		\end{enumerate}
\end{enumerate}

\noindent
These formal reviews are carried out as described in Ref. 1. At least three days before a formal review an informal review will be carried out, without any customer representative involved. The informal reviews are scheduled so that updated material can be altered to be delivered to the formal review. In addition to the informal reviews listed above the SDDD should also be reviewed in an informal review. 

\section{Test Cases}
	\subsection{Test Environments}
	All tests will be run in one or more of the following environments. 
		\begin{enumerate}
			\item JUnit 5.

			\item Web Browser (Google Chrome, Mozilla Firefox, Microsoft Edge).

			\item Operating System (Windows).

			\item Database System (MySQL).

			\item IDE (Eclipse).	
		\end{enumerate}
		
		\subsection{Test Types}
		
These are the types of tests that will be carried out throughout the project


\begin{itemize}
  \item \textbf{Unit test:} This is carried out by individual developers while they are coding and before the code is pushed to Github. This is done informally, and no documentation is needed.
  
  \item \textbf{Function test:} When a developer has pushed a new function to Github he/she should notify the test group that the function is ready to be tested by the test group, whom then run the tests and notes outcome in a spreadsheet. The results of the tests are then sent back to the developers. The tests that are perform can be found under appendix A. 
  
   \item \textbf{System test:} General tests for the whole system are carried out by the test group whom notes the test outcome in a spreadsheet. These are carried out after phase three and four. The tests that are perform can be found under appendix B. 
   
      \item \textbf{Regression test:} If a function that has already been tested is updated, the corresponding function test shall be redone. The tests that are performed can be found under appendix C. 
      
         \item \textbf{Acceptance test:} These tests are performed by the customer and is a combination of the tests stated above and the customers own tests.
\end{itemize}


\section{Requirements Checked in Reviews}
In reviews all requirements should be checked by the reviewers. The results of the review are then sent back to the respective group responsible for the document.

The following requirements (according to Ref. 2) are only verified through
reviews: 6.2.2, 7.0.1, 7.0.2, 7.0.3, 7.0.4, 7.0.5, 7.0.6, 7.0.7, 7.0.9, 7.0.10, 7.0.11, 7.0.12, 7.0.13, 8.1.1, 8.1.2, 8.1.3, 8.1.4, 8.1.5, 8.1.6.


\section{Appendices}
	\begin{itemize}
		\item \textbf{A.} Function Test Specification 
		\item \textbf{B.} System Test Specification 	
		\item \textbf{C.} Regression Test Specification 
		\end{itemize}
	
	\newpage
		\begin{flushleft}
		{\large \textbf{A Function Test Specification}}
		\end{flushleft}
		
		\subsection{Login and Logout}
		
		\begin{itemize}
  			\item \textbf{FT1:} User can login with correct username and password (JUnit). [Ref. 3. req. 6.1.1. 6.1.3, 6.1.6, 6,1,7. Ref. 2. req. 7.0.8]	
			
			\item \textbf{FT2:} User cannot login with non existing user(JUnit). [Ref. 3. req. 6.1.5]
			
			\item \textbf{FT3:} User cannot login with other users passwords. [Ref. 3. req. 6.1.5]
			
			\item \textbf{FT4:} User cannot login with foregoing password. [Ref. 3. req. 6.1.5]
				
			\item \textbf{FT5:} User can logout. [Ref. 3. req. 6.1.4 ]
			
  			\item \textbf{FT6:} User gets a new password from the server by e-mail if requested. [Ref. 2. req. 6.1.1]
  			
  			\item \textbf{FT7:} User gets a new password from the server by e-mail on account creation. [Ref. 2. req. 6.5.2]
  					
  			\item \textbf{FT8:} User can change password from logged-in state. [Ref. 2. req. 6.3.4] 
  						
  			\item \textbf{FT9:} User tries to change password from logged-in state but enters invalid new password. [Ref. 2. req. 6.3.5] 

  			\item \textbf{FT10:} User is shown relevant error message on failed login. [Ref. 3. req. 6.1.5. Ref. 2. req. 8.1.6] 
		\end{itemize}
		
		\subsection{Data}
		
		\begin{itemize}
		
  			\item \textbf{FT11:}  When changing password, it must adhere to password ruleset. [Ref. 2. req. 6.2.1, 6.3.4, 6.3.5]

		\end{itemize}
		
		\subsection{User}
		
		\begin{itemize}
  			\item \textbf{FT12:} Users can submit a time report. [Ref. 2. req. 6.3.1]
  			
  			\item \textbf{FT13:} Can see a summarized view of current user time reports. [Ref. 2. req. 6.3.2]
  			
  			\item \textbf{FT14:} Can edit a submitted report if it is not signed by Project leader. [Ref. 2. req. 6.3.3, 6.3.6]
  			
  			 \item \textbf{FT15:} Cannot edit a submitted report if it is signed by Project leader. [Ref. 2. req. 6.3.7]


		\end{itemize}
		
		\subsection{Project Leader}
		
			\begin{itemize}
			
			\item \textbf{FT16:} Can see a summarized view of users time reports. [Ref. 2. req. 6.4.1]
			
  			\item \textbf{FT17:} Can see all users and their roles. [Ref. 2. req. 6.4.2]

  			\item \textbf{FT18:} Can change roles on other users. [Ref. 2. req. 6.4.3, 6.4.7]
  			
  			\item \textbf{FT19:} Cannot change roles on themselves. [Ref. 2. req. 6.4.3]
  			
  			\item \textbf{FT20:} Cannot change other project leaders' roles. [Ref. 2. req. 6.4.3]
  			
  			\item \textbf{FT21:} Can sign time reports. [Ref. 2. req. 6.4.4, 6.4.6]
  			
  			\item \textbf{FT22:} Can reverse a signed time report making it editable again. [Ref. 2. req. 6.4.5]

		\end{itemize}
		
		\subsection{Administrator}
		
			\begin{itemize}
			
	
			\item \textbf{FT23:} Can access administration page. [Ref. 2. req. 6.5.1]			
			
  			\item \textbf{FT24:} Can add a new user. [Ref. 2. req. 6.5.2]
  		
  			\item \textbf{FT25:} Cannot add a new user with existing username. [Ref. 2. req. 6.5.3]

  			\item \textbf{FT26:} Can remove a user. [Ref. 2. req. 6.5.4]
  			
  			\item \textbf{FT27:} Can see a summarized view of users time reports. [Ref. 2. req. 6.5.5]  		  			
  			\item \textbf{FT28:} Can change the roles of all users. [Ref. 2. req. 6.5.6]
  			
  			\item \textbf{FT29:} Can change the roles of all project leaders. [Ref. 2. req. 6.5.6]
  			
  			\item \textbf{FT30:} Can view all users as a list. [Ref. 2. req. 6.5.7]

		\end{itemize}
		
		
		\newpage
		\begin{flushleft}
		{\large \textbf{B System Test Specification}}
		\end{flushleft}
		
		
		
		\begin{itemize}
		
  			\item \textbf{ST1:} Perform FT3 - FT30 in Google Chrome on Windows 10. 	
  			
  			\item \textbf{ST2:} Perform FT3 - FT30 in Mozilla Firefox on Windows 10. 
  			
  			\item \textbf{ST3:} Perform FT3 - FT30 in Microsoft Edge on Windows 10. 					
  			
  			
		\end{itemize}
		
		\newpage
		\begin{flushleft}
		{\large \textbf{C Regression Test Specification}}
		\end{flushleft}
			
		\begin{flushleft}
		Function tests are redone, as well as the following tests if an update is made to a specific function:
		\end{flushleft}
		
		\begin{itemize}
		
  			\item \textbf{RT1:} FT1
  			
			  					
		\end{itemize}
		
		\begin{flushleft}
			If ten or more functions are redone then the following regression test shall be executed:
			\end{flushleft}	
		
		\begin{itemize}			
  			
  			\item \textbf{RT2:} FT1 - FT30		

		\end{itemize}
		
		
			
		


%---------Document ends here-----------

\end{document}
