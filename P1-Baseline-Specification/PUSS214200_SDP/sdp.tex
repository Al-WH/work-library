\documentclass{article}
\usepackage[utf8]{inputenc}
\usepackage{fancyhdr}
\usepackage{graphicx}
\usepackage{hyperref}


% ---- Commands ------- %
\newcommand{\documentNumber}[1]{
    \LARGE  \textbf{ PUSP2142{#1} } \\
    \medskip
}
\newcommand{\documentVersion}[1]{
    v.{#1} \\
    \medskip
}
\newcommand{\documentTitle}[1]{
    \centerline{\rule{13cm}{0.4pt}}
    \bigskip \bigskip
    \LARGE {#1} \\
    \bigskip \bigskip
    \centerline{\rule{13cm}{0.4pt}}
}
\newcommand{\documentGroup}[1]{
    \bigskip \bigskip
    \LARGE Group {#1} \\
    \bigskip
}
\newcommand{\documentResponsible}[1]{
    \LARGE Responsible: {#1} \\
    \medskip
}
\newcommand{\documentAuthors}[1]{
    \LARGE Authors: {#1} \\
    \medskip    
}
\newcommand{\documentDate}[1]{
    \date {#1} 
}


% --- Header & Footer ---- %
\pagestyle{fancy}
\lhead{\leftmark}
\rhead{}
\rfoot{\thepage}
\cfoot{}
\lfoot{}


% ------------------------------------------------ #


\title {
    \documentNumber {01}    % Must be 2 digits
    \documentVersion {0.1}
    \documentTitle {Software development plan - SDP 123}
    \documentGroup {2}
    \documentResponsible {(PG) Projet-management Group}
    \documentAuthors {(PG) Projet-management Group}
    \documentDate {\today}
}

\begin{document}

\maketitle
\thispagestyle{empty}

\newpage

\tableofcontents

\newpage


\section{Introduction}
    This document describes the development model and the development plan for !!NAME!!, 
    which will be a system for time reporting and is based on \textit{Baseblock System}.
    !!NAME!! will be developed by students at LTH in the course 
    \textit{ETSF20 Programvaruutveckling för stora projekt}.

\section{Terminology}
    \renewcommand{\arraystretch}{1.7}  % Vertical padding for tables
    
    \begin{table}[h]
        \centering
        \begin{tabular}{| p{1.5cm} | p{9cm} |}
            \hline
                Baseblock system & This is the base system that is used in !!NAME!! \\
            \hline
                SG & System architecture Group \\
            \hline
                DG & Developer Group \\
            \hline
                QG & Quality control Group \\
            \hline
                PG & Project management Group \\
            \hline 
                CML & Configuration Management List, consists of all configuration units \\
            \hline            
                ECG & Error Control Group, consists of PG and SG \\
            \hline
                SDP & Software Development Plan \\
            \hline
                SDP & Software Development Plan \\
            \hline
                SRS & Software Requirements Specification \\
            \hline
                SVVS & Software Verification and Validation Specification \\
            \hline
                SVVI & Software Verification and Validation Instruction \\
            \hline
                STLDD & Software Top Level Design Document \\
            \hline
                SDDD & Software Detailed Design Document \\
            \hline
                SVVR & Software Verification and Validation Report \\
            \hline
                SSD & System Specification Document \\
            \hline
        \end{tabular}
    \end{table}

\section{Referenced documents}
    \begin{itemize}
        \item \href{https://google.github.io/styleguide/javaguide.html}{Google Java Style Guide}
        \item CML
    \end{itemize}
    

\section{Development model} %Assar och Victor
    The development model that is used in this project is the waterfall model. This
    means that the project is divided into four seperate phases where each phase depends
    on the previous one in a sequential manner. It is thus required that a phase is
    completed before the next one begins.
    \\ \\
    In every phase, there are several documents that must be produced and a phase
    is considered completed only once all documents required in the phase has reached baseline.
    To reach baseline, all documents of the phase must first pass an informal review, 
    followed by a formal review.
    
\section{Staff organisation} %Assar

    \subsection{Client}
    
    \subsection{Head of section}
    
    \subsection{Experts}
    
    \subsection{Examiner}
    
    \subsection{Project managment group}
        \begin{itemize}
            \item Coordinating the group effort
            \item Making a project schedule
            \item Ensuring that every individual has the information they need
            \item Authoring the following documents
                \begin{itemize}
                    \item SDP
                    \item SSD
                    \item PRF
                \end{itemize} 
        \end{itemize}
    
    \subsection{Software architecture group}
        \begin{itemize}
            \item Designing the software architecture
            \item Delegating work to DG and QG
            \item Coordinating the writing of the following documents
                \begin{itemize}
                    \item SRS
                    \item STLDD
                    \item SDDD
                \end{itemize}
        \end{itemize}
            
    \subsection{Development group}
        \begin{itemize}
            \item Design GUI for the software
            \item Develop the architecture that SAG designed
            \item Co-author the following documents
                \begin{itemize}
                    \item 
                \end{itemize}
        \end{itemize}
        
    
    \subsection{Quality control group}

\newpage

\section{Schedule}
        
    \subsection{Estimated work load}
        PG holds one hour long meeting every week. PG has also scheduled for a two hour time slot every wednesday where every group works on their respective task. To supplement this, the groups are expected to hold their own planing meetings and work sessions. Since the workload of each group varies between phases, it seemed more fitting to allow groups the flexibility of managing their own time. Group leaders have been tasked with making sure that coordination between groups is constant throughout the project. PG also estimated that approximately an hour every week will be spent on general discussions in the discord channel.
        
        \begin{table}[h]
            \centering
            \begin{tabular}{|l|c|c|}
                \hline
                    \textbf{Activity} & \textbf{Frequency/week} & \textbf{Duration (h)} \\
                \hline
                    Project group meeting & 1 & 1 \\
                 \hline
                    Project group work & 1 & 2 \\
                 \hline
                    Subgroup work session & 2 & 2*2 \\
                 \hline
                    Self studies & 1  & 1 \\
                 \hline
                    Discussions & 1 & 1 \\
                 \hline
                    Reviews/Expert meetings & 1 & 1 \\
                 \hline
                    \textbf{Total hours} & & \textbf{10} \\
                 \hline
            \end{tabular}
            \caption{Estimated time for each activity per person, per week}
        \end{table}
    
    \subsection{Estimated phase schedule}
        Table \ref{phasetable} illustrates the estimated start and end dates 
        for every phase as well as estimated hours spent per week, per person.

        \begin{table}[h]
            \centering
            \begin{tabular}{|c|c|c|c|c|}
                \hline
                    \textbf{Phase} & \textbf{Start} & \textbf{End} 3 & \textbf{Work days} & 
                    \textbf{Estimated hours/week} \\
                \hline
                    1 & 18/1 & 5/2 & 15 & 10 \\
                 \hline
                    2 & 08/2 & 19/2 & 10 & 10 \\
                 \hline
                    3 & 22/2 & 5/3  & 10 & 10 \\
                 \hline
                    4 & 8/3 & 19/3  & 10 & 15 \\
                 \hline
            \end{tabular}
            \caption{Estimated start and end for the phases}
            \label{phasetable}
        \end{table}
        


\section{Standards \& tools}    % Victor
    In order to make the development process as easy and straight forward as possible,
    the project group has agreed on several standards and tools to use. 
    \\ \\
    The standards should be followed by every member and consists of the following:
    \begin{itemize}
        \item The soure code should follow the
        \href{https://google.github.io/styleguide/javaguide.html}{Google Java Style Guide}.
        \item All comments, commits and pull-requests should be in english.
    \end{itemize}
    
    \subsection{Discord}
    Discord is used as the many communcation tool in the group. A server has been setup and is
    used for both messaging, working together and having project meetings.
    The server consistst of several voice channels and several text channels, each
    with its specific purpose, eg "Meetings" or "Developer Group".
    
    \subsection{Github \& Git}
    Git is used for collaboration between decouments and the project library (see \ref{project_library})
    and is hosted on Github. Git provides abilities that  makes certain action very
    easy, such as pulling and pushing updates to the working repository and creating pull-requests for merging.
    
    \subsection{Eclipse}
        Eclipse is the primary IDE that is used due to its familarity in
        the group and its huge span of different properties such as plugins.
    
        \subsubsection{Egit}
            Egit is a plugin for Eclipse that provides the tools needed for a git workflow 
            in Eclipse IDE.
            
        \subsubsection{TeXlipse}
            TeXlipse is a plugin that provides the tools needed to compile and preview
            latex files directly in Eclipse IDE.
    

\section{Configuration managment}   % Victor
    All changes to units in the CML must follow a certain procedure once the documents
    are in baseline. Git and Github are the tools that are used to handle configuration management.

    \subsection{Project library \label{project_library}}
        The project library consists of two seperate libraries: Document libray and Work library.
    
        \subsubsection{Document library}
            The document library consists of all configuration units that have reached baseline.
            The purpose of this library is that the customer or a reviewer at any point should
            be able to access these documents.
            This means that this library is initially empty, and documents are added as the project proceeds. In the end of phase 4, all units found in CML can be found in this library.

            
        \subsubsection{Work library}
            The work library contains everything all files that are required during the project.
            The library is divided into three different branches where each branch has a unique purpose.
            
            \begin{itemize}
                \item \textbf{development} \\
                This branch is where the all development is made and is used as a placement for all files related
                to the project, regardless of the documents status.
                Every member in the group has free write access to this branch.
                
                \item \textbf{review} \\
                Once the documents in the development branch are ready for a formal review,  they are moved
                into the \textbf{review} branch. This requires a pull-request to be made, which must be
                reviewed and accepted by a reviewer before it is merged into the branch.
                
                \item \textbf{master} \\
                Once the documents pass the formal reviews, they are merged into \textbf{master} branch,
                which consists of only files that have reached baseline.
                Once new documents are added to this branch, they are also placed in the \textbf{document library}.
                Note that not all files are copied to the document library, but only the documents specified above.
                
            \end{itemize}
        
    \subsection{Bug management}
    
    \subsection{Patch management}
        Once a configuration unit has reached baseline there are several steps
        that must be taken in order to make changes to it.
        These steps consists of the following:
        \begin{enumerate}
            \item Creating a error report that states what the
                    error is with the unit, in its current state.
            \item The error report is handed to the ECG who creates a 
                    status report. ECG then decides if the error is legitimate.
            \item If the error is deemed legitimate, then they
                    propose a solution and
                    decides whom shall be responsible to fix the problem.
                    Once the error is corrected, ECG must approve it
                    and then the unit version must be updated.
                    The status report is then closed.
            \item If ECG decides that the error is to be discarded, then
                    the problem and status report is closed.
                
        \end{enumerate}
        

    \subsection{Version naming \& update}
        
        
    


\section{Rules and guidelines} %Assar

% Reagera på pinnade meddelande.
% 
    
\section{Follow up and quality evaluation}
    To keep the quality of all documents as high as possible during the project,
    two reviews will be carried out and the end of every phase. One informal review
    followed by one formal review.
    
    \subsection{Informal Review}
        The informal review is performed within the project group and is meant to catch
        errors, bugs or mistakes in the documents. The purpose is to ensure as high quality
        as possible for the formal review and the philosophy is that if documents pass
        the informal review, they should also pass the formal review.
        
    \subsection{Formal Review}
        Once the documents have passed the informal review, they must pass the formal review,
        before they can reach baseline. During the formal review, and extern reviewer is
        given the documents at least 48 hours in advance so that he or she can prepare.
        During the actual review, the majority of the project group shall be present.
        The formal review can result in one of the four scenarios:
        \begin{enumerate}
            \item The documents are approved
            \item The documents are approved after certain modification.
            \item Modifications must be made followed by a re-review.
            \item The documents are \emph{not} approved and must be re-written,
                    followed by a new review.
        \end{enumerate}
    
    \subsection{Re-Review}
        An opportunity to do a review after certain modifications have been made.
        

Det ska finnas en del i projektplanen som beskriver hur uppföljning, t ex av
tidplanen, sker under projektet, samt vad som händer om arbetet inte verkar
gå enligt plan. Det ska också finnas en beskrivning av de rutiner som finns för
kvalitetsutvärdering under projektet.

    

\section{Risk anlysis}

I projektplanen ska även resultatet av en riskanalys för projektet presenteras.
Ange hur riskanalys utförts i projektet, samt de viktigaste riskerna som identi-
fierats. rapportera åtminstone följande för varje rapporteras risk: skattad san-
nolikhet (t ex, låg, medel, hög), skattad effekt (t ex, låg, medel, hög), möjliga
indikatorer på att risken förvekligas, samt exempel på lösningar om risken för-
verkligas.


\end{document}
